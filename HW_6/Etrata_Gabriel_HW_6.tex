\documentclass[12pt]{article}
\usepackage{geometry}
\geometry{letterpaper, margin=1in}
\usepackage{setspace}
%\doublespacing
\usepackage{color}
\usepackage{xcolor}
\usepackage{listings}
\usepackage{fancyhdr}
\usepackage[yyyymmdd,hhmmss]{datetime}
\usepackage{amsmath,amssymb,amsthm, mathtools, enumitem}
\usepackage{setspace}
\usepackage{enumitem}
\usepackage{url}
\doublespacing

% Solarized colour scheme for listings
\definecolor{solarized@base03}{HTML}{002B36}
\definecolor{solarized@base02}{HTML}{073642}
\definecolor{solarized@base01}{HTML}{586e75}
\definecolor{solarized@base00}{HTML}{657b83}
\definecolor{solarized@base0}{HTML}{839496}
\definecolor{solarized@base1}{HTML}{93a1a1}
\definecolor{solarized@base2}{HTML}{EEE8D5}
\definecolor{solarized@base3}{HTML}{FDF6E3}
\definecolor{solarized@yellow}{HTML}{B58900}
\definecolor{solarized@orange}{HTML}{CB4B16}
\definecolor{solarized@red}{HTML}{DC322F}
\definecolor{solarized@magenta}{HTML}{D33682}
\definecolor{solarized@violet}{HTML}{6C71C4}
\definecolor{solarized@blue}{HTML}{268BD2}
\definecolor{solarized@cyan}{HTML}{2AA198}
\definecolor{solarized@green}{HTML}{859900}

% Define C++ syntax highlighting colour scheme
\lstset{language=C++,
	basicstyle=\footnotesize\ttfamily,
	numbers=left,
	numberstyle=\footnotesize,
	tabsize=2,
	breaklines=true,
	escapeinside={@}{@},
	numberstyle=\tiny\color{solarized@base01},
	keywordstyle=\color{solarized@green},
	stringstyle=\color{solarized@cyan}\ttfamily,
	identifierstyle=\color{solarized@blue},
	commentstyle=\color{solarized@base01},
	emphstyle=\color{solarized@red},
	frame=single,
	rulecolor=\color{solarized@base2},
	rulesepcolor=\color{solarized@base2},
	showstringspaces=false
}

\begin{document}
	\raggedright{Gabriel Etrata} \hfill \raggedleft{Last Modified on \today\ at \currenttime}\\
	\raggedright{\textbf{MTH 4300 SMWA}, Spring 2018}\\%class
	\raggedright{Professor Ivan Matic, Baruch College}\\
	\textit{Homework 6} \\%topic
	\hrulefill\\
	\setlength\parindent{24pt} 

\textbf{Problem 2}: What is the output of \texttt{move\_assignment.cpp}? Provide a detailed explanation for your answer. \\
The main function calls heavyPrinting() which creates two MyClass objects, and outputs:\\
11\\
heavyPrinting() then assigns object2 to object1, calling the copy operator and outputs:\\
4\\
heavyPrinting() then creates another MyClass objects and outputs:\\
1\\
heavyPrinting() then creates assigns myFunction() to object3 and outputs:\\
1\\
myFunction() creates a Myclass object called temp, calling the copyFrom constructor and outputs:\\
4\\
The destructor gets called and outputs:\\
6\\
The copyFrom operator gets called and outputs:\\
5\\
The destructor gets called and outputs:\\
6\\
heavyPrinting() outputs:\\
7\\
The destructor gets called 3 times (3 destructed objects) and outputs:\\
666\\
Final output: 1141146567666


	

\end{document}
